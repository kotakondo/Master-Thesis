% $Log: abstract.tex,v $
% Revision 1.1  93/05/14  14:56:25  starflt
% Initial revision
% 
% Revision 1.1  90/05/04  10:41:01  lwvanels
% Initial revision
% 
%
%% The text of your abstract and nothing else (other than comments) goes here.
%% It will be single-spaced and the rest of the text that is supposed to go on
%% the abstract page will be generated by the abstractpage environment.  This
%% file should be \input (not \include 'd) from cover.tex.
Multiagent trajectory planning is a critical aspect of unmanned aerial vehicle (UAV) applications such as search and rescue missions, surveillance, and package delivery. However, developing a scalable and robust multiagent trajectory planner for UAVs poses significant challenges. Real-world deployments face difficulties such as communication delays and flying through dynamic environments, which can lead to undesired collisions.
\\

To address these challenges, we propose a decentralized, asynchronous multiagent trajectory planner called Robust MADER (RMADER). With centralized planners, each agent must listen to a single entity that plans all the trajectories, but this approach may lead to a single point of failure and depend heavily on communication with the central entity. In contrast, decentralized planners enable each agent to plan its own trajectory, making them inherently more scalable. Asynchronous planning allows each agent to trigger the planning step independently, without waiting at a synchronization barrier, making it more scalable than synchronous methods.
\\

However, decentralized, asynchronous multiagent trajectory planners are susceptible to communication delays, which could cause collisions. To overcome this issue, RMADER is designed to be robust to communication delays by introducing a delay-check step and a two-step trajectory-sharing scheme. RMADER guarantees safety by always keeping a collision-free trajectory and performing a delay check step, even under communication delay. 
\\

To evaluate RMADER, we performed extensive benchmark studies against state-of-the-art trajectory planners and flight experiments using a decentralized communication architecture called a mesh network with multiple UAVs in dynamic environments. The results demonstrate RMADER's robustness and capability to carry out collision avoidance in dynamic environments, outperforming existing state-of-the-art methods with a 100\% collision-free success rate.
\\

In summary, this thesis proposes a decentralized, asynchronous approach to multiagent trajectory planning approach for UAVs that is scalable and robust to communication delay. RMADER's promising results in both simulation and flight experiments contribute to the advancement of multiagent trajectory planning for UAVs.
